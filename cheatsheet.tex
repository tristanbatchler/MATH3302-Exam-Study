\documentclass{article}
\usepackage{amsmath,amsthm,amssymb, needspace, enumerate, listings, mathtools}

\newcounter{qcounter}
\newcommand{\showqcounter}{\theqcounter}
\newcommand{\question}{\vspace{5mm}\addtocounter{qcounter}{1}\vspace{5mm}{\bf Q\showqcounter: }}
\newcommand{\answer}{\vspace{5mm}{\bf A\showqcounter: }}

\setlength\parindent{0pt}

\begin{document}
\question Define Euler's Totient Function of a positive integer $n$ with words.

\answer The number of non-negative integers less than $n$ which are coprime to $n$.


\question Precicely describe how to calculate Euler's Totient Function of a positive integer $n$.

\answer First find the prime factorisation of $n$, call it 
$n = {p_1}^{e_1} \cdot {p_2}^{e_2} \cdots {p_k}^{e_k}$. Next calculate the following to find $\phi(n)$:
$$
    \prod_{i = 1}^k \left( {p_i}^{e_i - 1} \cdot (p_i - 1) \right)
$$

\question State Euler's Theorem.

\answer If $\gcd(a, n) = 1$, then $a^{\phi(n)} \equiv 1 \pmod n$.


\question State Fermat's Little Theorem.

\answer If $a$ is a positive integer and $p$ is prime, then $a^p \equiv 1 \pmod{p}$.


\question What does it mean for a sett of integers to be pairwise coprime?

\answer The greatest common divisor of all the elements is $1$.


\question State the Chinese Remainder Theorem.

\answer if $m_1, m_2, \cdots, m_r$ are pairwise coprime positive integers and $a_1, a_2, \cdots, a_r$ are integers, 
then the system of congruences 
\begin{align*}
x \equiv& a_1 \pmod{m_1} \\
x \equiv& a_2 \pmod{m_2} \\
  \vdots&\\
x \equiv& a_r \pmod{m_r}
\end{align*}
has a unique solution modulo $M := m1 \cdot m_2 \cdots m_r$ which is given by
$$
    x = \sum_{i=1}^r a_i M_i y_i \pmod{M}
$$
where $M_i := M / m_i$ and $y_i :\equiv M_i^{-1} \pmod{m_i}$ for $1 \leq i \leq r$. 


\question What is Kerckhoff's Assumption?

\answer Everything about a cryptographic system is public knowledge except for the key. In other words, the enemy knows the system.


\question Briefly explain the four levels of attacks on a cryptosystem, in ascending order of strength.

\answer \begin{enumerate}
  \item {\bf Ciphertext only:} Opponent has access to some ciphertext and might use statistical information to determine the corresponding plaintext.
  \item {\bf Known plaintext:} Opponent knows some plaintext-ciphertext pairs and uses it to gain more knowledge of the key.
  \item {\bf Chosen plaintext:} Opponent is temporarily able to encrypt to build a collection of known plaintext-ciphertext pairs. 
  \item {\bf Chosen plaintext-ciphertext:} Opponent is temporarily able to encrypt and decrypt.
\end{enumerate}


\question What is the plaintext and ciphertext space of a shift cipher?

\answer Both are the non-negative integers less than $26$.


\question Is a shift cipher a public or private system? What is/are the key(s)?

\answer Private. Key is the shift amount.


\question How do you encrypt a plaintext message with a shift cipher?

\answer Add the private key to the message modulo $26$. In other words, rotate the letter by the key amount.


\question How do you decrypt a ciphertext message encrypted with a shift cipher?

\answer Subtract the private key from the message modulo $26$. In other words, rotate the letter backwards by the key amount.


\question What are the best attacks for a shift cipher?

\answer Brute force (try every key) or analyse letter frequencies to determine the key.


\question What is the plaintext and ciphertext space for an affine cipher?

\answer Both are non-negative integers less than $26$.


\question Is an affine cipher a public or private system? What is/are the key(s)?

\answer Private. The key is a pair of non-negative integers less than $26$ where one of them is coprime to $26$.


\question How do you encrypt a plaintext message with an affine cipher?

\answer Suppose the key is $(a, b)$ where $a$ is coprime with $26$ and the message is $x$. To encrypt the message, calculate:
  $$
    y \equiv ax + b \pmod{26}
  $$


\question How do you decrypt a ciphertext message encrypted with an affine cipher?

\answer Suppose the key is $(a, b)$ where $a$ is coprime with $26$ and the encrypted message is $y$. The decrypt the message, calculate: 
  $$
    x \equiv a^{-1}(y - b) \pmod{26}
  $$

\question What are the best attacks for an affine cipher?

\answer Obtain two plaintext-ciphertext pairs and solve the resulting linear congruences for the key. Could also brute force.
  In otherwords, given $x_1, y_1, x_2, y_2$, solve the following for $a$ and $b$.
  \begin{align*}
    y_1 &\equiv ax_1 + b \pmod{26}\\
    y_2 &\equiv ax_2 + b \pmod{26}
  \end{align*}

  You can also just try all combinations of $a$ and $b$.

\question 

\answer 


\question 

\answer 


\question 

\answer 


\question 

\answer 


\question 

\answer 
\end{document}