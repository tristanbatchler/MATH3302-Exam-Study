\documentclass{article}
\usepackage{amsmath,amsthm,amssymb, needspace, enumerate, listings, mathtools}

\newcounter{qcounter}
\newcommand{\showqcounter}{\theqcounter}
\newcommand{\question}{\vspace{5mm}\addtocounter{qcounter}{1}\vspace{5mm}{\bf Q\showqcounter: }}
\newcommand{\answer}{\vspace{5mm}{\bf A\showqcounter: }}

\setlength\parindent{0pt}

\begin{document}
\question Define Euler's Totient Function of a positive integer $n$ with words.

\answer The number of non-negative integers less than $n$ which are coprime to $n$.


\question Precicely describe how to calculate Euler's Totient Function of a positive integer $n$.

\answer First find the prime factorisation of $n$, call it 
$n = {p_1}^{e_1} \cdot {p_2}^{e_2} \cdots {p_k}^{e_k}$. Next calculate the following to find $\phi(n)$:
$$
    \prod_{i = 1}^k \left( {p_i}^{e_i - 1} \cdot (p_i - 1) \right)
$$

\question State Euler's Theorem.

\answer If $\gcd(a, n) = 1$, then $a^{\phi(n)} \equiv 1 \pmod n$.


\question State Fermat's Little Theorem.

\answer If $a$ is a positive integer and $p$ is prime, then $a^p \equiv 1 \pmod{p}$.


\question What does it mean for a sett of integers to be pairwise coprime?

\answer The greatest common divisor of all the elements is $1$.


\question State the Chinese Remainder Theorem.

\answer if $m_1, m_2, \cdots, m_r$ are pairwise coprime positive integers and $a_1, a_2, \cdots, a_r$ are integers, 
then the system of congruences 
\begin{align*}
x \equiv& a_1 \pmod{m_1} \\
x \equiv& a_2 \pmod{m_2} \\
  \vdots&\\
x \equiv& a_r \pmod{m_r}
\end{align*}
has a unique solution modulo $M := m1 \cdot m_2 \cdots m_r$ which is given by
$$
    x = \sum_{i=1}^r a_i M_i y_i \pmod{M}
$$
where $M_i := M / m_i$ and $y_i :\equiv M_i^{-1} \pmod{m_i}$ for $1 \leq i \leq r$. 


\question 

\answer 


\question 

\answer 


\question 

\answer 


\question 

\answer 


\question 

\answer 


\question 

\answer 


\question 

\answer 


\question 

\answer 


\question 

\answer 


\question 

\answer 
\end{document}