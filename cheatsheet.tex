\documentclass{article}
\usepackage{amsmath,amsthm,amssymb, needspace, enumerate, listings, mathtools}

\newcounter{qcounter}
\newcommand{\showqcounter}{\theqcounter}
\newcommand{\question}{\vspace{5mm}\addtocounter{qcounter}{1}\vspace{5mm}{\bf Q\showqcounter: }}
\newcommand{\answer}{\vspace{5mm}{\bf A\showqcounter: }}

\setlength\parindent{0pt}

\begin{document}
\question Define Euler's Totient Function of a positive integer $n$ with words.

\answer The number of non-negative integers less than $n$ which are coprime to $n$.


\question Precicely describe how to calculate Euler's Totient Function of a positive integer $n$.

\answer First find the prime factorisation of $n$, call it 
$n = {p_1}^{e_1} \cdot {p_2}^{e_2} \cdots {p_k}^{e_k}$. Next calculate the following to find $\phi(n)$:
$$
    \prod_{i = 1}^k \left( {p_i}^{e_i - 1} \cdot (p_i - 1) \right)
$$

\question State Euler's Theorem.

\answer If $\gcd(a, n) = 1$, then $a^{\phi(n)} \equiv 1 \pmod n$.


\question State Fermat's Little Theorem.

\answer If $a$ is a positive integer and $p$ is prime, then $a^p \equiv 1 \pmod{p}$.


\question What does it mean for a sett of integers to be pairwise coprime?

\answer The greatest common divisor of all the elements is $1$.


\question State the Chinese Remainder Theorem.

\answer if $m_1, m_2, \cdots, m_r$ are pairwise coprime positive integers and $a_1, a_2, \cdots, a_r$ are integers, 
then the system of congruences 
\begin{align*}
x \equiv& a_1 \pmod{m_1} \\
x \equiv& a_2 \pmod{m_2} \\
  \vdots&\\
x \equiv& a_r \pmod{m_r}
\end{align*}
has a unique solution modulo $M := m_1 \cdot m_2 \cdots m_r$ which is given by
$$
    x = \sum_{i=1}^r a_i M_i y_i \pmod{M}
$$
where $M_i := M / m_i$ and $y_i :\equiv M_i^{-1} \pmod{m_i}$ for $1 \leq i \leq r$. 


\question What is Kerckhoff's Assumption?

\answer Everything about a cryptographic system is public knowledge except for the key. In other words, the enemy knows the system.


\question Briefly explain the four levels of attacks on a cryptosystem, in ascending order of strength.

\answer \begin{enumerate}
  \item {\bf Ciphertext only:} Opponent has access to some ciphertext and might use statistical information to determine the corresponding plaintext.
  \item {\bf Known plaintext:} Opponent knows some plaintext-ciphertext pairs and uses it to gain more knowledge of the key.
  \item {\bf Chosen plaintext:} Opponent is temporarily able to encrypt to build a collection of known plaintext-ciphertext pairs. 
  \item {\bf Chosen plaintext-ciphertext:} Opponent is temporarily able to encrypt and decrypt.
\end{enumerate}


\question What is the plaintext and ciphertext space of a shift cipher?

\answer Both are the non-negative integers less than $26$.


\question Is a shift cipher a public or private system? What is/are the key(s)?

\answer Private. Key is the shift amount.


\question How do you encrypt a plaintext message with a shift cipher?

\answer Add the private key to the message modulo $26$. In other words, rotate the letter by the key amount.


\question How do you decrypt a ciphertext message encrypted with a shift cipher?

\answer Subtract the private key from the message modulo $26$. In other words, rotate the letter backwards by the key amount.


\question What are the best attacks for a shift cipher?

\answer Brute force (try every key) or analyse letter frequencies to determine the key.


\question What is the plaintext and ciphertext space for an affine cipher?

\answer Both are non-negative integers less than $26$.


\question Is an affine cipher a public or private system? What is/are the key(s)?

\answer Private. The key is a pair of non-negative integers less than $26$ where one of them is coprime to $26$.


\question How do you encrypt a plaintext message with an affine cipher?

\answer Suppose the key is $(a, b)$ where $a$ is coprime with $26$ and the message is $x$. To encrypt the message, calculate:
  $$
    y \equiv ax + b \pmod{26}
  $$


\question How do you decrypt a ciphertext message encrypted with an affine cipher?

\answer Suppose the key is $(a, b)$ where $a$ is coprime with $26$ and the encrypted message is $y$. To decrypt the message, calculate: 
  $$
    x \equiv a^{-1}(y - b) \pmod{26}
  $$

\question What are the best attacks for an affine cipher?

\answer Obtain two plaintext-ciphertext pairs and solve the resulting linear congruences for the key. Could also brute force.
  In otherwords, given $x_1, y_1, x_2, y_2$, solve the following for $a$ and $b$.
  \begin{align*}
    y_1 &\equiv ax_1 + b \pmod{26}\\
    y_2 &\equiv ax_2 + b \pmod{26}
  \end{align*}

  You can also just try all combinations of $a$ and $b$.

\question What is the plaintext and ciphertext space for a mixed alphabet cipher?

\answer Both are non-negative integers less than $26$.


\question Is a mixed alphabet cipher a public or private system? What is/are the key(s)?

\answer Private. The key is a permutation of the set of non-negative integers less than $26$.


\question How do you encrypt a plaintext message with a mixed alphabet cipher?

\answer Suppose the key is a permutation of the set of non-negative integers less than $26$, $\pi$ and the message is $x$.
  To encrypt the message, calculate:
  $$
    y = \pi(x).
  $$


\question How do you decrypt a ciphertext message encrypted with a mixed alphabet cipher?

\answer Suppose the key is a permutation of the set of non-negative integers less than $26$, $\pi$ and the encrypted message is $y$.
To decrypt the message, calculate:
$$
  x = \pi^{-1}(y).
$$


\question What are the best attacks for a mixed alphabet cipher?

\answer Use statistical analysis on the known language's letter frequencies. Can also search for common digrams and trigrams 
  in the ciphertext which will correspond to things like ``th'', ``he'', ``in'', or ``the'', ``ing'', ``and'', etc. 
  Piece together the key bit by bit.


\question What is the plaintext and ciphertext space for a Vigen\`ere cipher?

\answer Both are strings of non-negative integers less than $26$.


\question Is a Vigen\`ere cipher a public or private system? What is/are the key(s)?

\answer Private. The key is a string of non-negative integers less than $26$.


\question How do you encrypt a plaintext message with a Vigen\`ere cipher?

\answer Suppose the key is a string of $m$ non-negative integers less than $26$, $k_1, k_2, \cdots k_m$.
  Further suppose the message is a string of $m$ non-negative integers less than $26$, $x_1, x_2, \cdots x_m$.
  To encrypt the message, calculate:
  \begin{align*}
    y_1 \equiv& x_1 + k_1 \pmod{26}\\
    y_2 \equiv& x_2 + k_2 \pmod{26}\\
              &\vdots\\
    y_m \equiv& x_m + k_m \pmod{26}
  \end{align*}

  In other words, treat each element of the key and message as its own shift cipher and encrypt it accordingly.


\question How do you decrypt a ciphertext message encrypted with a Vigen\`ere cipher?

\answer Suppose the key is a string of $m$ non-negative integers less than $26$, $k_1, k_2, \cdots k_m$.
  Further suppose the encrypted message is a string of $m$ non-negative integers less than $26$, $y_1, y_2, \cdots y_m$.
  To encrypt the message, calculate:
  \begin{align*}
    x_1 \equiv& y_1 - k_1 \pmod{26}\\
    x_2 \equiv& y_2 - k_2 \pmod{26}\\
              &\vdots\\
    x_m \equiv& y_m - k_m \pmod{26}
  \end{align*}

  In other words, treat each element of the key and encrypted message as its own shift cipher and decrypt it accordingly.

\question What are the best attacks for a Vigen\`ere cipher?

\answer Determine the key length, $m$, and break $m$ shift ciphers independently using frequency analysis.

\question Name and explain one method to determine the key length of a Vigen\`ere cipher.

\answer Kasiski's test. Find repeated trigrams in the ciphertext. If the key length is less than the number of occurrences of 
  a trigram, then its corresponding plaintext must have been encrypted with the same shift more than once (by the Pigeonhole Principle).

  We can then extract a common factor from the distances between the starting index of each repeated trigram to estimate the key length.


\question What is the index of coincidence of a language? How is it denoted?

\answer The probability of drawing two matching letters through random selection from a text in the given language.
  It is denoted with the greek letter $\varphi$ (phi).


\question How do you calculate the index of coincidence for a language with $n$ letters where the $i$th letter has 
  probability of occurring $p_i$?

\answer 
  $$
    \sum_{i=1}^n p_i^2
  $$


\question What is the index of coincidence approximately equal to for English?

\answer $0.0667$.


\question What is the index of coincidence equal to for a random language?

\answer $1 / 26 \approx 0.0384$.


\question Name and explain one method to determine the key length of a Vigen\`ere cipher.

\answer Friedman's first method. Suppose your guess for the key length is $m$. Extract every $m$th letter from the ciphertext 
  and calculate the index of coincidence of the extracted text. 
  If it's close to the index of coincidence for English, it's probably the correct key length.


\question Name and explain one method to determine the key length of a Vigen\`ere cipher.

\answer Friedman's second method. Calculate the index of coincidence for the entire ciphertext ($\varphi_T$) and use the measure 
  of how ``flat'' is it compared with English ($\varphi_L$) and random text ($\varphi_0$) to estimate the key length.
  $$
    m \approx \frac{\varphi_L - \varphi_0}{\varphi_T - \varphi_0}
  $$


\question What is the plaintext and ciphertext space for a Hill cipher?

\answer Strings of non-negative integers less than $26$.


\question Is a Hill cipher a public or private system? What is/are the key(s)?

\answer Private. The key is a invertible matrix of non-negative integers less than $26$.


\question How do you encrypt a plaintext message with a Hill cipher?

\answer Suppose the key is an invertible $m \times m$ matrix $K$ and $x$ is a message of length $m$. 
  To encrypt the message, calculate:
  $$
    y \equiv K x \pmod{26}
  $$


\question How do you decrypt a ciphertext message encrypted with a Hill cipher?

\answer Suppose the key is an invertible $m \times m$ matrix $K$ and $y$ is an encrypted message of length $m$. 
To decrypt the message, calculate:
$$
  x \equiv K^{-1} y \pmod{26}
$$


\question What are the best attacks for a Hill cipher?

\answer Obtain $m$ plaintext-ciphertext pairs each of length $m$.
  Solve the resulting linear congruences for $K$:
  \begin{align*}
    y_1 \equiv& K x_1 \pmod{26}\\
    y_2 \equiv& K x_2 \pmod{26}\\
              & \vdots\\
    y_m \equiv& K x_m \pmod{26}
  \end{align*}


\question How can we easily calculate Euler's Totient function of $n$ where $n$ is the product of two primes? In other words, 
how can we arrive at a simple expression for $\phi(p q)$?.

\answer We note that the prime factorisation of $n$ is $p^1 q^1$ so:
$$
  \phi(n) = \left[p^{1 - 1} (p - 1)\right] \cdot \left[q^{1 - 1} (q - 1)\right] = (p - 1) (q - 1)
$$


\question What is the plaintext and ciphertext space for RSA?

\answer Both are non-negative integers less than the product of two primes.


\question Is RSA a public or private system? What is/are the key(s)?

\answer Public. 
  \begin{itemize}
    \item The {\bf public} key is $(n, b)$ where $n$ is the product of two {\bf distinct} large primes $p$ and $q$ and 
      $b$ is a randomly chosen positive integer less than $\phi(n)$ ($\phi$ is Euler's Totient function) which is 
      invertible modulo $\phi(n)$.
    \item The {\bf private} key is $(a, p, q)$ where $a$ is $b^{-1} \pmod{\phi(n)}$ and $p$ and $q$ are 
      the primes whose product makes $n$.
  \end{itemize}


\question How do you encrypt a plaintext message with RSA?

\answer Suppose our public key is $(n, b)$ and the message is $x$. To encrypt the message, calculate:
  $$
    y \equiv x^b \pmod{n}.
  $$


\question How do you decrypt a ciphertext message with RSA?

\answer Suppose our private key is $(a, p, q)$ so $n = p q$ and our encrypted message is $y$. To decrypt the message, caclulate:
  $$
    x \equiv y^a \pmod{n}.
  $$


\question What are the best attacks for RSA?

\answer 
  \begin{itemize}
    \item If our private key has one prime much smaller than the other, we can easily factorise $n$.
    \item If our private key has primes roughly the same size, we can also easily factorise $n$.
    \item If the message to encrypt, $x$, is so small that $y := x^b \pmod{n}$ is less than $n$, then you can simply take the $b$th root 
      of $y$ over the reals to get back $x$.
    \item If the message to encrypt is predictable, simply try encrypting guesses and see if the resulting ciphertext matches.
    \item If the public exponent, $b$ is so small that the same plaintext message is likely to be encrypted with $b$ or more different keys, 
      you can recover the plaintext using the Chinese Remainder Theorem.
  \end{itemize}


\question Why is it important to sign and verify signatures of messages encrypted with public key cryptography?

\answer Because the public key of every sender is known, anybody can use the public key to encrypt any message and send to anybody. 
  You need some claim of authenticity in the messages you send and receive.


\question How can a sender sign their RSA-encrypted message?

\answer Attach a signature to their encrypted message, which is a hashed plaintext encrypted with their private key rather than the public one.
  In other words, if the hashed message is $f(x)$ and the private exponent is $a$, the attaches the following signature to the ciphertext:
  $$
    s \equiv \left(f(x)\right)^a \pmod{n}
  $$


\question How can a receiver verify the signature of an RSA-encrypted message?

\answer Decrypt the signature, $s$, with the intended sender's public exponent, $b$, and see if it results in the hashed decrypted message $f(x)$.
  In other words, verify the following holds:
  $$
    s^b \equiv f(x) \pmod{n}
  $$


\question How could a malicious sender send an encrypted message with RSA claiming to be somebody they're not.

\answer They obtain the public intended encrypted message and the signature and look for a collision in the hash function. 
  They can then send their malicious encrypted message along with the original sender's signature and nobody will be able to 
  detect foul play.


\question What is the plaintext and ciphertext space for ElGamal?

\answer Both are non-negative integers less than some large prime.


\question Is ElGamal a public or private system? What is/are the key(s)?

\answer Public. 
  \begin{itemize}
    \item The {\bf public} key is a set of three integers, $(p, \alpha, \beta)$ where:
      \begin{itemize}
        \item $p$ is a large prime,
        \item $\alpha$ is a positive integer less than $p$, which, when repeatedly squared, 
          covers the entire set of positive integers less than $p$,
        \item $\beta \equiv \alpha^a \pmod{p}$ where $a$ is a random integer between $2$ and $p - 2$.
      \end{itemize}
    \item The {\bf private} key is $a$ which is the random integer between $2$ and $p - 2$ from the 
      definition of $\beta$ in the public key.
  \end{itemize}

\question How do you encrypt a plaintext message with ElGamal?

\answer Suppose our public key is $p, \alpha, \beta$ and the message to send is $x$. 
  To encrypt the message, choose a random integer $d$ between $2$ and $p - 2$ and caclulate:
  $$
    (c_1, c_2) \equiv \left(\alpha^d,\, \beta^d x \right) \pmod{p}
  $$


\question How do you decrypt a ciphertext message with ElGamal?

\answer Suppose our private key is $a$ and the encrypted message is $(c_1, c_2)$. To decrypt the message, caclulate:
  $$
    x \equiv (c_1^a)^{-1}c_2 \pmod{p}
  $$


\question What are the best attacks for ElGamal?

\answer If we obtain one plaintext-ciphertext pair and another ciphertext which was encrypted using the same random exponent 
  $d$, we can recover the plaintext corresponding to the second ciphertext. \\
  
  Suppose we know $x_1$ encrypts to $(y_1, y_2)$ and we wish to find the value of some $x_2$ that maps to $(c_1, c_2)$. \\
  
  We first note the value of $\beta^d$ as follows:
  \begin{align*}
    \beta^d \equiv y_2 x_1^{-1} \pmod{p}
  \end{align*}

  Next we use this expression to calculate the value of $x_2$ as follows:
  \begin{align*}
    x_2 &\equiv \left(\beta^d\right)^{-1} c_2 \pmod{p}\\
        &\equiv \left( y_2 x_1^{-1} \right)^{-1} c_2 \pmod{p}\\
        &\equiv y_2^{-1} x_1 c_2 \pmod{p}
  \end{align*}


\question Explain how a man-in-the-middle attack works where the sender, opponent, and receiver are named Alice, Eve, and Bob respectively.

\answer 
  \begin{itemize}
    \item Alice wants to encrypt a message and send it to Bob.
    \item Eve can listen to Alice's requests to the public key directory and inject her own communicatios to Alice and Bob.
    \item Alice requests Bob's public key from the directory.
    \item Eve silently sends Alice {\emph her} public key instead of Bob's.
    \item Alice encrypts her message with Eve's public key instead of Bob's and sends it through Eve with intent of arriving to Bob.
    \item Eve decrypts the message with her private key and reads it.
    \item Eve then re-encrypts the message with Bob's public key and sends it to Bob.
    \item Neither Alice nor Bob know Eve has read the message.
  \end{itemize}


\question What does PKI stand for and what does it aim to achieve?

\answer Public Key Infrastructure. It provides a means of trusting the authenticity of senders and receivers of publicly encrypted messages.


\question What is one method of certification in PKI? Roughly how does it work?

\answer Certificate Authorities. A publicly trusted certificate authority (CA) signs and publishes people's public keys verifying they are 
  who they say they are. If you trust the CA, you can trust a public key belongs to somebody if it's been signed by that CA.


\question What is one method of certification in PKI? Roughly how does it work?

\answer Web of Trust. There is no centralised certificate authority, but instead, end users are all encouraged to meet each other and personally 
  verify each other's identity and sign their public keys in person.


\question What is the difference between a synchronous and asynchronous stream cipher?

\answer 
  \begin{itemize}
    \item {\bf Synchronous:} The keystream is independent of the plaintext.
    \item {\bf Asynchronous:} The keystream uses the plaintext to generate the element of the stream.
  \end{itemize}

  
\question What does it mean for a stream cipher to be periodic with period $d$?

\answer The keystream repeats itself after $d$ elements.


\question What is the plaintext and ciphertext space for the Autokey cipher?

\answer Both are non-negative integers less than 26.


\question Is the Autokey cipher a public or private system? What is/are the key(s)?

\answer Private. The key is a non-negative integer less than 26 is called the ``seed'' and the rest of the keystream continues to take on 
  the values from the plaintext message but does not include the last element of the plaintext message.


\question How do you encrypt a plaintext message with the Autokey cipher?

\answer Add the key stream to the message modulo $26$. In other words, add the seed to the first letter of the message and then 
  add the first letter of the message to the next letter of the message and so on until you add the second last letter of the message 
  to the last letter of the message


\question How do you decrypt a plaintext message with the Autokey cipher?

\answer Subtract the key stream from the message modulo $26$.


\question Is the Autokey cipher synchronous or asynchronous?

\answer Synchronous.


\question Is the Autokey cipher periodic? If so, what is the period?

\answer It's non-periodic.


\question What are the best attacks for the Autokey cipher?

\answer There are only 26 initial seeds. Just try them all and look for something that looks like English when you decrypt using that seed.


\question What does it mean for a linear feedback shift register to be $m$-stage?

\answer The seed must include a pair of bitstrings of length $m$ where the neither is ending with a $0$.


\question What is the plaintext and ciphertext space for a linear feedback shift register?

\answer Both are binary digits (bits).


\question Is an $m$-stage linear feedback shift register a public or private system? What is/are the key(s)?

\answer Private. The key is a pair of strings of $m$ bits which is called the ``seed''. The next element of the keystream is a linear 
  combination of the previous $m$ bits of the keystream. For example, if the seed consists of the following two bit strings:
  \begin{align*}
    l_1, l_2, \cdots, l_m \\
    c_0, c_1, \cdots, c_{m-1}
  \end{align*}

  then the $(m + i)$th element of the keystream is given by:
  $$
    l_{m + i} = c_0 l_{i} + c_1 l_{i + 1} + \cdots + c_{m - 1} l_{i + m - 1}
  $$


\question How do you encrypt a plaintext message with a linear feedback shift register?

\answer Add the keystream to the message.


\question How do you decrypt a plaintext message with a linear feedback shift register?

\answer Subtract the keystream from the message.


\question Is a linear feedback shift register synchronous or asynchronous?

\answer Synchronous.


\question Is an $m$-stage linear feedback shift register periodic? If so, what is the period?

\answer Yes. A good choice of starting seed can give a period of $2^m - 1$.


\question What are the best attacks for an $m$-stage linear feedback shift register?

\answer If you have $2m$ plaintext-ciphertext pairs, you can easily compute the first $2m$ keystream 
  bits by adding each plaintext-ciphertext pair (i.e. $l_i \equiv x_i + y_i \pmod{2}$).\\

  From there, solve $m$ linear equations for $c_0, c_1, \cdots c_{m-1}$.
  \begin{align*}
    l_{m + 1} =& c_0 l_1 + c_1 l_2 + \cdots + c_{m - 1} l_m \\
    l_{m + 2} =& c_0 l_2 + c_2 l_3 + \cdots + c_{m - 1} l_{m + 1}\\
              &\vdots\\
    l_{2m}    =& c_0 l_m + c_2 l_{m + 1} + \cdots + c_{m - 1} l_{2m - 1}
  \end{align*}

  And then we will have obtained the seed (i.e. the key) to the cipher.


\question Name and label each component of Enigma.

\answer
  \begin{itemize}
    \item The keyboard
    \item The plugboard $S$
    \item The rotors $L, M, N$
    \item The reversing drum $R$
    \item The glowlamps
  \end{itemize}


\question What does the plugboard do in Enigma?

\answer Allows up to $6$ pairs of letters to be swapped as configured by the user. 


\question What do the three rotors do in Enigma? How do they work?

\answer
  \begin{itemize}
    \item Each rotor arbitrarily permutes the letters coming in. Not swapping, you can't simply run a letter through the rotors twice to get back the same thing.
    \item Each rotor has its permutation hardwired. 
    \item The first rotor spins after every key press.
    \item The second rotor spins every time the first rotor has made a full revolution (i.e. every $26$ key presses).
    \item The third rotor spins every time the second rotor has made a full revolution (i.e. every $26 \times 26$ key presses).
  \end{itemize}


\question What doe reversing drum do in Enigma?

\answer Arranges the $26$ letters into $16$ pairs and swaps each pair of letters so that the input is never equal to the output. 
  This swapping permutation is hardwired into the drum.


\question Describe the path of a signal from its initial position on the keyboard through to its final position on the glowlamps.

\answer
  \begin{enumerate}
    \item Signal runs through the switch board to potentially get swapped to a signal representing a different letter.
    \item Signal runs through each rotor, being arbitrarily permuted as it does.
    \item Signal runs through the reversing drum, being swapped with a different letter (not itself).
    \item Signal runs back through each rotor again in reverse, being arbitrarily permuted again as it does.
    \item Signal runs back through the plugboard to potentially swap whichever letter comes in with a different letter. 
  \end{enumerate}

\end{document}