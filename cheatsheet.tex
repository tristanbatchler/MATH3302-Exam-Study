\documentclass{article}
\usepackage{amsmath,amsthm,amssymb, needspace, enumerate, listings, mathtools}

\newcounter{qcounter}
\newcommand{\showqcounter}{\theqcounter}
\newcommand{\question}{\vspace{5mm}\addtocounter{qcounter}{1}\vspace{5mm}{\bf Q\showqcounter: }}
\newcommand{\answer}{\vspace{5mm}{\bf A\showqcounter: }}

\setlength\parindent{0pt}

\begin{document}
\question Define Euler's Totient Function of a positive integer $n$ with words.

\answer The number of non-negative integers less than $n$ which are coprime to $n$.


\question Precicely describe how to calculate Euler's Totient Function of a positive integer $n$.

\answer First find the prime factorisation of $n$, call it 
$n = {p_1}^{e_1} \cdot {p_2}^{e_2} \cdots {p_k}^{e_k}$. Next calculate the following to find $\phi(n)$:
$$
    \prod_{i = 1}^k \left( {p_i}^{e_i - 1} \cdot (p_i - 1) \right)
$$

\question State Euler's Theorem.

\answer If $\gcd(a, n) = 1$, then $a^{\phi(n)} \equiv 1 \pmod n$.


\question State Fermat's Little Theorem.

\answer If $a$ is a positive integer and $p$ is prime, then $a^p \equiv 1 \pmod{p}$.


\question What does it mean for a sett of integers to be pairwise coprime?

\answer The greatest common divisor of all the elements is $1$.


\question State the Chinese Remainder Theorem.

\answer if $m_1, m_2, \cdots, m_r$ are pairwise coprime positive integers and $a_1, a_2, \cdots, a_r$ are integers, 
then the system of congruences 
\begin{align*}
x \equiv& a_1 \pmod{m_1} \\
x \equiv& a_2 \pmod{m_2} \\
  \vdots&\\
x \equiv& a_r \pmod{m_r}
\end{align*}
has a unique solution modulo $M := m1 \cdot m_2 \cdots m_r$ which is given by
$$
    x = \sum_{i=1}^r a_i M_i y_i \pmod{M}
$$
where $M_i := M / m_i$ and $y_i :\equiv M_i^{-1} \pmod{m_i}$ for $1 \leq i \leq r$. 


\question What is Kerckhoff's Assumption?

\answer Everything about a cryptographic system is public knowledge except for the key. In other words, the enemy knows the system.


\question Briefly explain the four levels of attacks on a cryptosystem, in ascending order of strength.

\answer \begin{enumerate}
  \item {\bf Ciphertext only:} Opponent has access to some ciphertext and might use statistical information to determine the corresponding plaintext.
  \item {\bf Known plaintext:} Opponent knows some plaintext-ciphertext pairs and uses it to gain more knowledge of the key.
  \item {\bf Chosen plaintext:} Opponent is temporarily able to encrypt to build a collection of known plaintext-ciphertext pairs. 
  \item {\bf Chosen plaintext-ciphertext:} Opponent is temporarily able to encrypt and decrypt.
\end{enumerate}


\question What is the plaintext and ciphertext space of a shift cipher?

\answer Both are the non-negative integers less than $26$.


\question Is a shift cipher a public or private system? What is/are the key(s)?

\answer Private. Key is the shift amount.


\question How do you encrypt a plaintext message with a shift cipher?

\answer Add the private key to the message modulo $26$. In other words, rotate the letter by the key amount.


\question How do you decrypt a ciphertext message encrypted with a shift cipher?

\answer Subtract the private key from the message modulo $26$. In other words, rotate the letter backwards by the key amount.


\question What are the best attacks for a shift cipher?

\answer Brute force (try every key) or analyse letter frequencies to determine the key.


\question What is the plaintext and ciphertext space for an affine cipher?

\answer Both are non-negative integers less than $26$.


\question Is an affine cipher a public or private system? What is/are the key(s)?

\answer Private. The key is a pair of non-negative integers less than $26$ where one of them is coprime to $26$.


\question How do you encrypt a plaintext message with an affine cipher?

\answer Suppose the key is $(a, b)$ where $a$ is coprime with $26$ and the message is $x$. To encrypt the message, calculate:
  $$
    y \equiv ax + b \pmod{26}
  $$


\question How do you decrypt a ciphertext message encrypted with an affine cipher?

\answer Suppose the key is $(a, b)$ where $a$ is coprime with $26$ and the encrypted message is $y$. To decrypt the message, calculate: 
  $$
    x \equiv a^{-1}(y - b) \pmod{26}
  $$

\question What are the best attacks for an affine cipher?

\answer Obtain two plaintext-ciphertext pairs and solve the resulting linear congruences for the key. Could also brute force.
  In otherwords, given $x_1, y_1, x_2, y_2$, solve the following for $a$ and $b$.
  \begin{align*}
    y_1 &\equiv ax_1 + b \pmod{26}\\
    y_2 &\equiv ax_2 + b \pmod{26}
  \end{align*}

  You can also just try all combinations of $a$ and $b$.

\question What is the plaintext and ciphertext space for a mixed alphabet cipher?

\answer Both are non-negative integers less than $26$.


\question Is a mixed alphabet cipher a public or private system? What is/are the key(s)?

\answer Private. The key is a permutation of the set of non-negative integers less than $26$.


\question How do you encrypt a plaintext message with a mixed alphabet cipher?

\answer Suppose the key is a permutation of the set of non-negative integers less than $26$, $\pi$ and the message is $x$.
  To encrypt the message, calculate:
  $$
    y = \pi(x).
  $$


\question How do you decrypt a ciphertext message encrypted with a mixed alphabet cipher?

\answer Suppose the key is a permutation of the set of non-negative integers less than $26$, $\pi$ and the encrypted message is $y$.
To decrypt the message, calculate:
$$
  x = \pi^{-1}(y).
$$


\question What are the best attacks for a mixed alphabet cipher?

\answer Use statistical analysis on the known language's letter frequencies. Can also search for common digrams and trigrams 
  in the ciphertext which will correspond to things like ``th'', ``he'', ``in'', or ``the'', ``ing'', ``and'', etc. 
  Piece together the key bit by bit.


\question What is the plaintext and ciphertext space for a Vigen\`ere cipher?

\answer Both are strings of non-negative integers less than $26$.


\question Is a Vigen\`ere cipher a public or private system? What is/are the key(s)?

\answer Private. The key is a string of non-negative integers less than $26$.


\question How do you encrypt a plaintext message with a Vigen\`ere cipher?

\answer Suppose the key is a string of $m$ non-negative integers less than $26$, $k_1, k_2, \cdots k_m$.
  Further suppose the message is a string of $m$ non-negative integers less than $26$, $x_1, x_2, \cdots x_m$.
  To encrypt the message, calculate:
  \begin{align*}
    y_1 \equiv& x_1 + k_1 \pmod{26}\\
    y_2 \equiv& x_2 + k_2 \pmod{26}\\
              &\vdots\\
    y_m \equiv& x_m + k_m \pmod{26}
  \end{align*}

  In other words, treat each element of the key and message as its own shift cipher and encrypt it accordingly.


\question How do you decrypt a ciphertext message encrypted with a Vigen\`ere cipher?

\answer Suppose the key is a string of $m$ non-negative integers less than $26$, $k_1, k_2, \cdots k_m$.
  Further suppose the encrypted message is a string of $m$ non-negative integers less than $26$, $y_1, y_2, \cdots y_m$.
  To encrypt the message, calculate:
  \begin{align*}
    x_1 \equiv& y_1 - k_1 \pmod{26}\\
    x_2 \equiv& y_2 - k_2 \pmod{26}\\
              &\vdots\\
    x_m \equiv& y_m - k_m \pmod{26}
  \end{align*}

  In other words, treat each element of the key and encrypted message as its own shift cipher and decrypt it accordingly.

\question What are the best attacks for a Vigen\`ere cipher?

\answer Determine the key length, $m$, and break $m$ shift ciphers independently using frequency analysis.

\question Name and explain one method to determine the key length of a Vigen\`ere cipher.

\answer Kasiski's test. Find repeated trigrams in the ciphertext. If the key length is less than the number of occurrences of 
  a trigram, then its corresponding plaintext must have been encrypted with the same shift more than once (by the Pigeonhole Principle).

  We can then extract a common factor from the distances between the starting index of each repeated trigram to estimate the key length.


\question What is the index of coincidence of a language? How is it denoted?

\answer The probability of drawing two matching letters through random selection from a text in the given language.
  It is denoted with the greek letter $\varphi$ (phi).


\question How do you calculate the index of coincidence for a language with $n$ letters where the $i$th letter has 
  probability of occurring $p_i$?

\answer 
  $$
    \sum_{i=1}^n p_i^2
  $$


\question What is the index of coincidence approximately equal to for English?

\answer $0.0667$.


\question What is the index of coincidence equal to for a random language?

\answer $1 / 26 \approx 0.0384$.


\question Name and explain one method to determine the key length of a Vigen\`ere cipher.

\answer Friedman's first method. Suppose your guess for the key length is $m$. Extract every $m$th letter from the ciphertext 
  and calculate the index of coincidence of the extracted text. 
  If it's close to the index of coincidence for English, it's probably the correct key length.


\question Name and explain one method to determine the key length of a Vigen\`ere cipher.

\answer Friedman's second method. Calculate the index of coincidence for the entire ciphertext ($\varphi_T$) and use the measure 
  of how ``flat'' is it compared with English ($\varphi_L$) and random text ($\varphi_0$) to estimate the key length.
  $$
    m \approx \frac{\varphi_L - \varphi_0}{\varphi_T - \varphi_0}
  $$


\question What is the plaintext and ciphertext space for a Hill cipher?

\answer Strings of non-negative integers less than $26$.


\question Is a Hill cipher a public or private system? What is/are the key(s)?

\answer Private. The key is a invertible matrix of non-negative integers less than $26$.


\question How do you encrypt a plaintext message with a Hill cipher?

\answer Suppose the key is an invertible $m \times m$ matrix $K$ and $x$ is a message of length $m$. 
  To encrypt the message, calculate:
  $$
    y \equiv K x \pmod{26}
  $$


\question How do you decrypt a ciphertext message encrypted with a Hill cipher?

\answer Suppose the key is an invertible $m \times m$ matrix $K$ and $y$ is an encrypted message of length $m$. 
To decrypt the message, calculate:
$$
  x \equiv K^{-1} y \pmod{26}
$$


\question What are the best attacks for a Hill cipher?

\answer Obtain $m$ plaintext-ciphertext pairs each of length $m$.
  Solve the resulting linear congruences for $K$:
  \begin{align*}
    y_1 \equiv& K x_1 \pmod{26}\\
    y_2 \equiv& K x_2 \pmod{26}\\
              & \vdots\\
    y_m \equiv& K x_m \pmod{26}
  \end{align*}


\question How can we easily calculate Euler's Totient function of $n$ where $n$ is the product of two primes? In other words, 
how can we arrive at a simple expression for $\phi(p q)$?.

\answer We note that the prime factorisation of $n$ is $p^1 q^1$ so:
$$
  \phi(n) = \left[p^{1 - 1} (p - 1)\right] \cdot \left[q^{1 - 1} (q - 1)\right] = (p - 1) (q - 1)
$$


\question What is the plaintext and ciphertext space for RSA?

\answer Both are non-negative integers less than the product of two primes.


\question Is RSA a public or private system? What is/are the key(s)?

\answer Public. 
  \begin{itemize}
    \item The {\bf public} key is $(n, b)$ where $n$ is the product of two {\bf distinct} large primes $p$ and $q$ and 
      $b$ is a randomly chosen positive integer less than $\phi(n)$ ($\phi$ is Euler's Totient function) which is 
      invertible modulo $\phi(n)$.
    \item The {\bf private} key is $(a, p, q)$ where $a$ is $b^{-1} \pmod{\phi(n)}$ and $p$ and $q$ are 
      the primes whose product makes $n$.
  \end{itemize}


\question How do you encrypt a plaintext message with RSA?

\answer Suppose our public key is $(n, b)$ and the message is $x$. To encrypt the message, calculate:
  $$
    y \equiv x^b \pmod{n}.
  $$


\question How do you decrypt a plaintext message with RSA?

\answer Suppose our private key is $(a, p, q)$ so $n = p q$ and our encrypted message is $y$. To decrypt the message, caclulate:
  $$
    x \equiv y^a \pmod{n}.
  $$


\question What are the best attacks for RSA?

\answer 
  \begin{itemize}
    \item If our private key has one prime much smaller than the other, we can easily factorise $n$.
    \item If our private key has primes roughly the same size, we can also easily factorise $n$.
    \item If the message to encrypt, $x$, is so small that $y := x^b \pmod{n}$ is less than $n$, then you can simply take the $b$th root 
      of $y$ over the reals to get back $x$.
    \item If the message to encrypt is predictable, simply try encrypting guesses and see if the resulting ciphertext matches.
    \item If the public exponent, $b$ is so small that the same plaintext message is likely to be encrypted with $b$ or more different keys, 
      you can recover the plaintext using the Chinese Remainder Theorem.
  \end{itemize}


\question Why is it important to sign and verify signatures of messages encrypted with public key cryptography?

\answer Because the public key of every sender is known, anybody can use the public key to encrypt any message and send to anybody. 
  You need some claim of authenticity in the messages you send and receive.


\question How can a sender sign their RSA-encrypted message?

\answer Attach a signature to their encrypted message, which is a hashed plaintext encrypted with their private key rather than the public one.
  In other words, if the hashed message is $f(x)$ and the private exponent is $a$, the attaches the following signature to the ciphertext:
  $$
    s \equiv \left(f(x)\right)^a \pmod{n}
  $$


\question How can a receiver verify the signature of an RSA-encrypted message?

\answer Decrypt the signature, $s$, with the intended sender's public exponent, $b$, and see if it results in the hashed decrypted message $f(x)$.
  In other words, verify the following holds:
  $$
    s^b \equiv f(x) \pmod{n}
  $$


\question How could a malicious sender send an encrypted message with RSA claiming to be somebody they're not.

\answer They obtain the public intended encrypted message and the signature and look for a collision in the hash function. 
  They can then send their malicious encrypted message along with the original sender's signature and nobody will be able to 
  detect foul play.


\question What is the plaintext and ciphertext space for ElGamal?

\answer Both are non-negative integers less than some large prime.


\question Is ElGamal a public or private system? What is/are the key(s)?

\answer Public. 
  \begin{itemize}
    \item The {\bf public} key is 
    \item The {\bf private} key is 
  \end{itemize}

\question How do you encrypt a plaintext message with ElGamal?

\answer ... To encrypt the message, caclulate:
$$
blah
$$


\question How do you decrypt a plaintext message with ElGamal?

\answer ... To decrypt the message, caclulate:
$$
blah
$$


\question 

\answer 


\question 

\answer 


\question 

\answer 


\question 

\answer 


\question 

\answer 


\question 

\answer 


\question 

\answer 


\question 

\answer 


\question 

\answer 


\question 

\answer 


\question 

\answer 
\end{document}